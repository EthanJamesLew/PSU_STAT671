\documentclass{article}[12pt]

% useful packages
\usepackage{fullpage}
\usepackage{amsmath,amssymb,amsthm,amsfonts}
\usepackage{graphicx}
\usepackage{enumerate}
\usepackage{algorithm,algorithmic}
\usepackage{xcolor}
\usepackage{bbm}
\usepackage{url}
\usepackage{caption,subcaption}

% theorem type environments
\newtheorem{thm}{Theorem}
\newtheorem{prop}{Proposition}
\newtheorem{lemma}{Lemma}
\newtheorem{cor}{Corollary}
\newtheorem{defn}{Definition}
\newtheorem{assump}{Assumption}
\newtheorem{example}{Example}
\newtheorem{conjecture}{Conjecture}

% frequently used symbols
\newcommand{\bE}{\mathbb{E}}
\newcommand{\bP}{\mathbb{P}}
\newcommand{\bQ}{\mathbb{Q}}
\newcommand{\bR}{\mathbb{R}}
\newcommand{\bS}{\mathbb{S}}
\newcommand{\bN}{\mathbb{N}}
\newcommand{\bZ}{\mathbb{Z}}
\newcommand{\sC}{{\mathcal C}} 
\newcommand{\sD}{{\mathcal D}} 
\newcommand{\sE}{{\mathcal E}} 
\newcommand{\sF}{{\mathcal F}} 
\newcommand{\sL}{{\mathcal L}} 
\newcommand{\sH}{{\mathcal H}} 
\newcommand{\sN}{{\mathcal N}} 
\newcommand{\sO}{{\mathcal O}} 
\newcommand{\sP}{{\mathcal P}} 
\newcommand{\sR}{{\mathcal R}} 
\newcommand{\sS}{{\mathcal S}}
\newcommand{\sU}{{\mathcal U}} 
\newcommand{\sX}{{\mathcal X}} 
\newcommand{\sY}{{\mathcal Y}} 
\newcommand{\sZ}{{\mathcal Z}}

% operators
\newcommand{\sign}{\mathop{\mathrm{sign}}}
\newcommand{\supp}{\mathop{\mathrm{supp}}} % support
\newcommand{\argmin}{\operatornamewithlimits{arg\ min}}
\newcommand{\argmax}{\operatornamewithlimits{arg\ max}}
\newcommand{\dist}{\operatorname{dist}}
\newcommand{\tr}{\text{tr}}
\newcommand{\vecop}{\text{vec}}
\newcommand{\st}{\operatorname{s.t.}}
\newcommand{\cut}{\setminus}
\newcommand{\ra}{\rightarrow}
\newcommand{\ind}[1]{\mathbbm{1}\left\{#1\right\}} 
\newcommand{\given}{\ | \ }

% grouping operators
\newcommand{\brac}[1]{\left[#1\right]}
\newcommand{\set}[1]{\left\{#1\right\}}
\newcommand{\abs}[1]{\left\lvert #1 \right\rvert}
\newcommand{\paren}[1]{\left(#1\right)}
\newcommand{\norm}[1]{\left\|#1\right\|}
\newcommand{\ip}[2]{\left\langle #1,#2 \right\rangle}

% code commands
\newcommand{\matlab}{\textsc{Matlab }}
\newcommand{\algname}[1]{\textnormal{\textsc{#1}}}

% header command
\newcommand{\homework}[4]{
    \pagestyle{myheadings}
    \thispagestyle{plain}
    \newpage
    \setcounter{page}{1}
    \setlength{\headsep}{10mm}
    \noindent
    \begin{center}
    \framebox{
        \vbox{\vspace{2mm}
            \hbox to 6.28in { {\bf STAT 671: Statistical Learning I
            \hfill Fall 2019} }
        \vspace{4mm}
        \hbox to 6.28in { {\Large \hfill Homework #1 \hfill} }
        \vspace{2mm}
        \hbox to 6.28in { \Large \hfill Due: #2 \hfill }
        \vspace{2mm}
        \hbox to 6.28in { {\it Student Name: #3} \hfill {\it Professor Name: #4}}
        \vspace{2mm}}
   }
   \end{center}
   \markboth{Homework #1}{Homework #1}
   \vspace*{4mm}
}

\begin{document}

\homework{1}{October 14, 2019}{Ethan Lew}{Bruno Jedynak}

\section{A Simple Classifier}

Given a set of observations associated with labels, $(x,y) \in \sX \times \{ \pm 1\}$, a binary classifier, $f : \sX \ra \{\pm 1 \}$, can be defined with two steps
 
 \begin{enumerate}
\item Create a function $h: \sX  \ra \bR$ .
\item Threshold: choose $u \in \bR$,

\begin{equation}
g(x) = \left\{\begin{array}{c}{1 \text { if } f(x) \geq u} \\ {-1 \text { if } f(x)<u}\end{array}\right..
\end{equation}
 \end{enumerate}
 
 Begin by computing the means of the two classes in feature space,
 
 \begin{equation}
 \mathbf{c}_{+}=\frac{1}{m_{+}} \sum_{\left\{i | y_{i}=+1\right\}} \mathbf{x}_{i}
 \end{equation}

\begin{equation}
\mathbf{c}_{-}=\frac{1}{m_{-}} \sum_{\left\{i | y_{i}=-1\right\}} \mathbf{x}_{i}
\end{equation}

A point on the boundary between the positive and negative labels is assumed to lie of the midpoint of the line between $\mathbf{c}_+$ and $\mathbf{c}_-$, $\mathbf{c}=\left(\mathbf{c}_{+}+\mathbf{c}_{-}\right) / 2$ . Further, the boundary is assumed to be orthogonal to the vector $\mathbf{w}=\mathbf{c}_{+}-\mathbf{c}_{-}$. Accordingly, a classifier function can be defined as,

\begin{equation}
\begin{aligned}
y &=\operatorname{sgn}\left\langle\left(\mathbf{x}-\left(\mathbf{c}_{+}+\mathbf{c}_{-}\right) / 2\right),\left(\mathbf{c}_{+}-\mathbf{c}_{-}\right)\right\rangle \\
&= \operatorname{sgn}\left(\left\langle\mathbf{x}, \mathbf{c}_{+}\right\rangle-\left\langle\mathbf{x}, \mathbf{c}_{-}\right\rangle+  \frac{1}{2} \left\langle \mathbf{c}_-, \mathbf{c}_{-}\right\rangle - \frac{1}{2} \left\langle \mathbf{c}_+, \mathbf{c}_{+}\right\rangle  \right)
\end{aligned}
\end{equation}

If a kernel, $k: \sH \times \sH \ra \bR$ is replaced by the inner product,

\begin{equation}
y=\operatorname{sgn}\left(\frac{1}{m_{+}} \sum_{\left\{i | y_{i}=+1\right\}} k\left(x, x_{i}\right)-\frac{1}{m_{-}} \sum_{\left\{i | y_{i}=-1\right\}} k\left(x, x_{i}\right)+b\right),
\end{equation}

where

\begin{equation}
b =\frac{1}{2}\left(\frac{1}{m_{-}^{2}} \sum_{\left\{(i, j) | y_{i}=y,=-1\right\}} k\left(x_{i}, x_{j}\right)-\frac{1}{m_{+}^{2}} \sum_{\left\{(i, j) | y_{i}=y_{i}=+1\right\}} k\left(x_{i}, x_{j}\right)\right).
\end{equation}




\end{document}