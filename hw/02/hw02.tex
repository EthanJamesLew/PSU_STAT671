\documentclass{article}[12pt]

% useful packages
\usepackage{fullpage}
\usepackage{amsmath,amssymb,amsthm,amsfonts}
\usepackage{graphicx}
\usepackage{enumerate}
\usepackage{algorithm,algorithmic}
\usepackage{xcolor}
\usepackage{bbm}
\usepackage{url}
\usepackage{caption,subcaption}

% theorem type environments
\newtheorem{thm}{Theorem}
\newtheorem{prop}{Proposition}
\newtheorem{lemma}{Lemma}
\newtheorem{cor}{Corollary}
\newtheorem{defn}{Definition}
\newtheorem{assump}{Assumption}
\newtheorem{example}{Example}
\newtheorem{conjecture}{Conjecture}

% frequently used symbols
\newcommand{\bE}{\mathbb{E}}
\newcommand{\bP}{\mathbb{P}}
\newcommand{\bQ}{\mathbb{Q}}
\newcommand{\bR}{\mathbb{R}}
\newcommand{\bS}{\mathbb{S}}
\newcommand{\bN}{\mathbb{N}}
\newcommand{\bZ}{\mathbb{Z}}
\newcommand{\sC}{{\mathcal C}} 
\newcommand{\sD}{{\mathcal D}} 
\newcommand{\sE}{{\mathcal E}} 
\newcommand{\sF}{{\mathcal F}} 
\newcommand{\sL}{{\mathcal L}} 
\newcommand{\sH}{{\mathcal H}} 
\newcommand{\sN}{{\mathcal N}} 
\newcommand{\sO}{{\mathcal O}} 
\newcommand{\sP}{{\mathcal P}} 
\newcommand{\sR}{{\mathcal R}} 
\newcommand{\sS}{{\mathcal S}}
\newcommand{\sU}{{\mathcal U}} 
\newcommand{\sX}{{\mathcal X}} 
\newcommand{\sY}{{\mathcal Y}} 
\newcommand{\sZ}{{\mathcal Z}}

% operators
\newcommand{\sign}{\mathop{\mathrm{sign}}}
\newcommand{\supp}{\mathop{\mathrm{supp}}} % support
\newcommand{\argmin}{\operatornamewithlimits{arg\ min}}
\newcommand{\argmax}{\operatornamewithlimits{arg\ max}}
\newcommand{\dist}{\operatorname{dist}}
\newcommand{\tr}{\text{tr}}
\newcommand{\vecop}{\text{vec}}
\newcommand{\st}{\operatorname{s.t.}}
\newcommand{\cut}{\setminus}
\newcommand{\ra}{\rightarrow}
\newcommand{\ind}[1]{\mathbbm{1}\left\{#1\right\}} 
\newcommand{\given}{\ | \ }

% grouping operators
\newcommand{\brac}[1]{\left[#1\right]}
\newcommand{\set}[1]{\left\{#1\right\}}
\newcommand{\abs}[1]{\left\lvert #1 \right\rvert}
\newcommand{\paren}[1]{\left(#1\right)}
\newcommand{\norm}[1]{\left\|#1\right\|}
\newcommand{\ip}[2]{\left\langle #1,#2 \right\rangle}

% code commands
\newcommand{\matlab}{\textsc{Matlab }}
\newcommand{\algname}[1]{\textnormal{\textsc{#1}}}

% header command
\newcommand{\homework}[4]{
    \pagestyle{myheadings}
    \thispagestyle{plain}
    \newpage
    \setcounter{page}{1}
    \setlength{\headsep}{10mm}
    \noindent
    \begin{center}
    \framebox{
        \vbox{\vspace{2mm}
            \hbox to 6.28in { {\bf STAT 671: Statistical Learning I
            \hfill Fall 2019} }
        \vspace{4mm}
        \hbox to 6.28in { {\Large \hfill Homework #1 \hfill} }
        \vspace{2mm}
        \hbox to 6.28in { \Large \hfill Due: #2 \hfill }
        \vspace{2mm}
        \hbox to 6.28in { {\it Student Name: #3} \hfill {\it Professor Name: #4}}
        \vspace{2mm}}
   }
   \end{center}
   \markboth{Homework #1}{Homework #1}
   \vspace*{4mm}
}

\begin{document}

\homework{2}{October 28, 2019}{Ethan Lew}{Dr. Bruno Jedynak}

\section{Kernels}
\begin{enumerate}
\item Let $(x,y) \in \mathbb{R}^+ \times \mathbb{R}^+$, where $\mathbb{R}^+=\{x \in \mathbb{R};x \geq 0\}$, the ``french positive\rq\rq{} real numbers. 
\begin{enumerate}
\item 
Verify that 
$$\min(x,y) = \int_0^\infty \mathbb{I}_{t\leq x} \mathbb{I}_{t\leq y} dt$$
where  
$$\mathbb{I}_A =  \left\{
\begin{tabular}{ll}
1 & \mbox{ if A is true}\\
0 & \mbox{otherwise}
\end{tabular}
\right.
$$ 

The $\operatorname{min}$ function can be expressed as the piecewise defined function,
\begin{equation}
\operatorname{min}(x,y)=\left\{\begin{array}{ll}{x} & {x \le y} \\ {y} & {x > y}\end{array}\right. .
\end{equation}
Evaluate the integral over both intervals. For $x \le y$,
\begin{equation}
\begin{aligned}
\int_{0}^{\infty} \mathbb{I}_{t \leq x} \mathbb{I}_{t \leq y} d t &= \int_{0}^{x}  d t \\
&= x .
\end{aligned}
\end{equation}
For $x > y$,
\begin{equation}
\begin{aligned}
\int_{0}^{\infty} \mathbb{I}_{t \leq x} \mathbb{I}_{t \leq y} d t &= \int_{0}^{y}  d t \\
&= y .
\end{aligned}
\end{equation}
Thus, the equivalence holds.


\item Use the previous question to show that $K(x,y)=\min(x,y)$ is a pd kernel over $\mathbb{R}^+$.


\begin{thm}{\textbf{Aronzsajn's Theorem}}
$k$ is a positive definite kernel on the set of $\mathcal X$ if and only if there exists a Hilbert space $\mathcal H$ and a mapping
\begin{equation}
\phi: \mathcal X \rightarrow \mathcal H,
\end{equation}
such that, for any $x, x' \in \mathcal X$:
\begin{equation}
k(x, x') = \langle \phi(x), \phi(x') \rangle_{\mathcal H}.
\end{equation}
\label{thm:aron}
\end{thm}

Consider the function $\phi:\mathbb R^+ \rightarrow \mathcal H$,  where $\mathcal H$ is a Hilbert space,
\begin{equation}
\mathcal H = \{f_x(t) = \mathbb{I}_{t \le x} | x \in \mathbb R^+\},
\end{equation}
endowed with inner product,
\begin{equation}
\langle f_x(t), f_y(t) \rangle = \int_{0}^{\infty}f_x(t)  f_y(t) d t.
\end{equation}
From the previous problem,
\begin{equation}
\langle f_x(t), f_y(t) \rangle = \operatorname{min}(x,y).
\end{equation}
By Theorem \ref{thm:aron}, $\text{min}(x,y)$ is the inner product of a reproducing kernel Hilbert space, meaning that can also be a positive definite kernel.


\item Show that $\max(x,y)$ is not a pd kernel over  $\mathbb{R}^+$. 

\begin{defn}{\textbf{Positive Definite Kernel}}
A kernel $k : \mathcal X \times \mathcal X \rightarrow \mathbb R$ is said to be positive definite if for any sequence $a_1,a_2,...,a_m \in \mathbb R$ and $x_1, x_2,...,x_m \in \mathcal X$,
\begin{equation}
\sum_{i=1}^{m} \sum_{i=1}^{m} a_i a_j k(x_i, x_j) \ge 0.
\end{equation}
\label{def:pd}
\end{defn}

Consider the sequence $a_1=1, a_2=-1$ and $x_1=0, x_2=1$,
\begin{equation}
\sum_{i=1}^{2} \sum_{i=1}^{2} a_i a_j k(x_i, x_j) = 0 -1 -1 + 1 = -1 \not \ge 0.
\end{equation}
From Definition \ref{def:pd}, the function $\text{max}(x,y)$ fails the condition, and is not positive definite.
\end{enumerate}
\item Consider a probability space $(\Omega,\mathcal{A},P)$
\begin{enumerate}
\item Define for any two events $A$ and $B$, 
$$K_1(A,B)=P(A \cap B)$$
where $A \cap B$ is the intersection between the events A and B 
Verify that $K_1$ is positive definite. Hint: $P(A)=E[\mathbb{I}_A]$

First acknowledge that,
\begin{equation}
\mathbb I _{A \cap B} = \mathbb{I}_A \mathbb{I}_B.
\end{equation}
Consequently, 
\begin{equation}\label{equ:event}
\begin{aligned} 
\mathbb E \left[ \mathbb I_{A \cap B} \right] &= \mathbb E \left[ \mathbb I_{A} \mathbb I_{B} \right]\\
&= \int_0^\infty \mathbb I_{A} \mathbb I_{B} f(t)
 dt. \end{aligned}
\end{equation}
Thus, for some density function, $f(t)$, the following Hilbert space $\mathcal H$ can be constructed,
\begin{equation}
\mathcal H_f = \left\{g_A = \mathbb I_A \sqrt{f} : A \in \mathcal A \right\},
\end{equation}
endowed with an inner product
\begin{equation}
\langle g_A, g_B\rangle = \int_{-\infty}^{\infty} g_A(t) g_B (t) dt.
\end{equation}
This is equivalent to Equation \ref{equ:event}. By Theorem \ref{thm:aron}, the kernel function is positive definite.


\item Define for any two events $A$ and $B$, 
$$K_2(A,B)=P(A \cap B)-P(A)P(B)$$
Verify that $K_2$ is positive definite. 


\end{enumerate}
\end{enumerate}
\section{Kernels and RKHS}

\begin{defn}{\textbf{Reproducing Kernel Hilbert Space}}
Let $\mathcal X$ be a nonempty set and by $\mathcal H$ a Hilbert space of functions $f: \mathcal X \rightarrow \mathbb R$. Then $\mathcal H$ is called a reproducing kernel Hilbert space endowed with the inner product $\langle.,.\rangle$ if there exists $k: \mathcal X \times \mathcal X \rightarrow \mathbb R$ with the following properties.
\begin{enumerate}
\item $k$ has the property 
\begin{equation}
\langle f, k(x, .) \rangle = f(x), \forall f \in \mathcal H.
\end{equation}
\item $k$ spans $\mathcal H$.
\end{enumerate}
\end{defn}

\begin{enumerate}
\item Define the RKHS  over $\mathbb{R}^d$
$$K(x,y)=x^Ty+c$$
where $c>0$. 
\begin{enumerate}
\item
What is the RKHS associated with the kernel $K$? no proof is required. 
\begin{equation}
\mathcal{H}=\left\{f_{v, v_{0}} ; f_{v, v_{0}}(x)=v^{T} x+\frac{v_{0}}{c} \middle| v \in \mathbb R^d, v_0 \in \mathbb R \right\}.
\end{equation}
\item 
What is the inner product in this RKHS? no proof required.  
\begin{equation}
\langle f_{v, v_0}, f_{w, w_0}\rangle = v^T w+\frac{v_0 w_0}{c}.
\end{equation}
\item 
Verify the reproducing property

To verify the reproducing property, it is required to show that
\begin{equation} \label{equ:repro}
\left\langle f_{v,v_0}, K(x, .)\right\rangle, \forall f_{v,v_0} \in \mathcal{H}.
\end{equation}
\begin{equation}
\begin{aligned}\left\langle f_{v,v_0}, K(x, .)\right\rangle &= v^Tx+\frac{c v+0}{c} \\
&= v^Tx+v_0 \\
&=f(x).\end{aligned}
\end{equation}
\end{enumerate}
\item Define the RKHS  over $\mathbb{R}^d$
$$K(x,y)=(x^Ty)^2$$
where $c>0$. 
The RKHS associated with the kernel $K$ is $\{f_S;f_S(x)=x^T S x\}$ where $S$ is a symmetric $(d,d)$ matrix. The inner product is
\begin{equation} \label{equ:frob}
<f_{S_1},f_{S_2}>=<S_1,S_2>_F
\end{equation}
\begin{enumerate}
\item
Verify the reproducing property. 

To verify the reproducing property, it is required to show that
\begin{equation} \label{equ:repro}
\langle f_S, K(x, .)\rangle= f_S(x), \forall f_S \in \mathcal{H}.
\end{equation}
Re-express $K$ into the form,
\begin{equation}
\begin{aligned} K(x, y)&=\left(x^{T} y\right)^{2} \\ 
&= \left(x^{T} y\right) \left(x^{T} y\right)\\
&= \left(x^{T} y\right)^T \left(x^{T} y\right) \\
&= y^Txx^Ty .\\
 \end{aligned}
\end{equation}
Is is evident that $x x^T$ is a symmetric matrix in $\mathbb R^{d \times d}$, as  $x x^T =\left(x x^T\right)^T$. Thus,
\begin{equation}\label{equ:ks}
\begin{aligned} K(x, y)&= y^T S_x y,
 \end{aligned}
\end{equation}
such that $S_x = x x^T$. Now, using Equation \ref{equ:frob}, evaluate the LHS of the reproducing property Equation \ref{equ:repro},
\begin{equation}
\begin{aligned}
\left\langle f_{S}, K(x, .)\right\rangle&= \langle S, S_x \rangle_F \\
&= \operatorname{tr}\left( S S_x \right)\\
&= \operatorname{tr}\left( Sx x^T \right).
\end{aligned}
\end{equation}
Now, evaluate the RHS of the reproducing property equation,
\begin{equation}
\begin{aligned}
f_{S}(x)&=x^{T} S x \\
&=\operatorname{tr}\left( S x x^T \right).
\end{aligned}
\end{equation}
Thus, the reproducing property holds.



\item 
Why do we require that $S$ is symmetric?

In order for the proposed set of functions and kernel to be a RKHS, it is required that the reproducing property and that $K$ spans $\mathcal H$,
\begin{equation}
\mathcal{H}=\operatorname{span}\{K(x, \cdot) | x \in \mathcal{X}\}.
\end{equation}
It was shown in Equation \ref{equ:ks} that $S_x = S_x^T$, so the kernel can only span the space $\mathcal H$ is the matrix $S$ is symmetric.

\end{enumerate}
\item Define the RKHS  over $\mathbb{R}^d$
$$K(x,y)=(x^Ty+c)^2$$
where $c>0$. 
\begin{enumerate}
\item
What is the RKHS associated with the kernel $K$? no proof is required. 

The kernel can be seen as a linear combination of the previous two kernels,
\begin{equation}
\begin{aligned}
K(x, y)&=\left(x^{T} y+c\right)^{2} \\
&= \left( x^T y\right)^2 + 2cx^Ty+c^2 \\
&=  \left( x^T y\right)^2  + 2c \left( x^Ty + \frac{c}{2} \right).
\end{aligned}
\end{equation}

A RKHS guess would be,
\begin{equation}
\mathcal{H}=\left\{f_{S,v, v_{0}} ; f_{v, v_{0}}(x)= x^TSx+v^{T} x+\frac{v_{0}}{c} | S = S^T \in \mathbb R^{d\times d}, v \in \mathbb{R}^{d}, v_{0} \in \mathbb{R}\right\}
\end{equation}
\item 
What is the inner product in this RKHS? no proof required.  
\begin{equation}
\langle f_{S_1, v, v_0}, f_{S_2, w, w_0} \rangle = \langle S_1, S_2 \rangle_F + 2c \left( v^T w + \frac{2v_0w_0}{c}\right).
\end{equation}
\item 
Verify the reproducing property
\end{enumerate}
\end{enumerate}
\section{Fisher kernel} 
Let\footnote{If you are not familiar with probabilistic models, do not panic! come to see me.} $\theta \in \mathbb{R}$ be a parameter and let $p_\theta$ be a probabilistic model (i.e a point mass function or a density) over a set $\mathcal{X}$ indexed by $\theta$. Let $\theta_0 \in \mathbb{R}$ be a specific value for $\theta$.

Let us define the Fisher score at $x \in \mathcal{X}$ as
\begin{equation}
\phi(x,\theta_0) = \frac{\delta}{\delta \theta} \ln p_\theta(x) \mbox{ evaluated at } \theta=\theta_0
\end{equation}
assuming that this quantity exists. 

Define $I(\theta)$, the Fisher information associated with the parameter $\theta$, i.e., 
\begin{equation}
I(\theta)=E[\phi^2(X,\theta)]
\end{equation}
where $E$ stands for expectation and $X$ is a random variable with distribution $p_\theta$. 

The Fisher kernel is then 
\begin{equation}
k(x,x')=\frac{\phi(x,\theta_0)\phi(x',\theta_0)}{I(\theta_0)}
\end{equation}
where 
\begin{enumerate}
\item Verify that $k(.,.)$ is a positive definite kernel over $\mathcal{X}$
\item Consider the following model: $x \in \{0,1\}$, $X \sim Bernoulli(\theta)$, $0 < \theta < 1$, that is
\begin{equation}
p_\theta(x)=\theta^x(1-\theta)^{(1-x)} 
\end{equation}
We recall that in this case $E[X]=\theta$ and $Var[X]=E[(X-\theta)^2]=\theta(1-\theta)$

Compute $k(x,x')$

Hint: you will find $$k(x,x')=\frac{(x-\theta_0)(x'-\theta_0)}{\theta_0(1-\theta_0)}$$
\item
Assume now $x=(x_1,x_2)$ with $x_1 \in \{0,1\}$ and $x_2 \in \{0,1\}$. 
We consider the following model where $X=(X_1,X_2)$, $X_1$ and $X_2$ are independent with the same $Bernoulli(\theta)$ distribution. 
Compute $k(x,x')$. 
\end{enumerate}
 
\end{document}